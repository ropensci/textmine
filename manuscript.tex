\documentclass[author-year, review, 11pt]{components/elsarticle} %review=doublespace preprint=single 5p=2 column
%%% Begin My package additions %%%%%%%%%%%%%%%%%%%
\usepackage[hyphens]{url}
\usepackage{lineno} % add
  \linenumbers % turns line numbering on
\bibliographystyle{elsarticle-harv}
\biboptions{sort&compress} % For natbib
\usepackage{graphicx}
\usepackage{booktabs} % book-quality tables
%% Redefines the elsarticle footer
\makeatletter
\def\ps@pprintTitle{%
 \let\@oddhead\@empty
 \let\@evenhead\@empty
 \def\@oddfoot{\it \hfill\today}%
 \let\@evenfoot\@oddfoot}
 \def\tightlist{}
\makeatother

% A modified page layout
\textwidth 6.75in
\oddsidemargin -0.15in
\evensidemargin -0.15in
\textheight 9in
\topmargin -0.5in
%%%%%%%%%%%%%%%% end my additions to header

\usepackage[T1]{fontenc}
\usepackage{lmodern}
\usepackage{amssymb,amsmath}
\usepackage{ifxetex,ifluatex}
\usepackage{fixltx2e} % provides \textsubscript
% use upquote if available, for straight quotes in verbatim environments
\IfFileExists{upquote.sty}{\usepackage{upquote}}{}
\ifnum 0\ifxetex 1\fi\ifluatex 1\fi=0 % if pdftex
  \usepackage[utf8]{inputenc}
\else % if luatex or xelatex
  \usepackage{fontspec}
  \ifxetex
    \usepackage{xltxtra,xunicode}
  \fi
  \defaultfontfeatures{Mapping=tex-text,Scale=MatchLowercase}
  \newcommand{\euro}{€}
\fi
% use microtype if available
\IfFileExists{microtype.sty}{\usepackage{microtype}}{}
\usepackage{longtable}
\ifxetex
  \usepackage[setpagesize=false, % page size defined by xetex
              unicode=false, % unicode breaks when used with xetex
              xetex]{hyperref}
\else
  \usepackage[unicode=true]{hyperref}
\fi
\hypersetup{breaklinks=true,
            bookmarks=true,
            pdfauthor={},
            pdftitle={rOpenSci tools for textmining open source science literature},
            colorlinks=true,
            urlcolor=blue,
            linkcolor=magenta,
            pdfborder={0 0 0}}
\urlstyle{same}  % don't use monospace font for urls
\setlength{\parindent}{0pt}
\setlength{\parskip}{6pt plus 2pt minus 1pt}
\setlength{\emergencystretch}{3em}  % prevent overfull lines
\setcounter{secnumdepth}{0}
% Pandoc toggle for numbering sections (defaults to be off)
\setcounter{secnumdepth}{0}
% Pandoc header



\begin{document}
\begin{frontmatter}

  \title{rOpenSci tools for textmining open source science literature}
    \author[cstar]{Scott Chamberlain\corref{c1}}
   \ead{scott(at)ropensci.org} 
   \cortext[c1]{Corresponding author}
      \address[cstar]{rOpenSci, Museum of Paleontology, University of California, Berkeley,
CA, USA}
  
  \begin{abstract}
  Corresponding Author:
  
  Scott Chamberlain
  
  rOpenSci, Museum of Paleontology, University of California, Berkeley,
  CA, USA
  
  Email address:
  \href{mailto:scott@ropensci.org}{\nolinkurl{scott@ropensci.org}}
  
  \newpage
  
  Background. xxxx.
  
  Methods. xxxx.
  
  Results. xxxx.
  
  Discussion. xxxx.
  \end{abstract}
  
 \end{frontmatter}


\newpage

\section{Introduction}\label{introduction}

Explosion of digital context online. But we need tools to facilitate
textmining (i.e., rOpenSci)

Why text mine it.

How to text mine it.

Here, we introduce and give examples for textmining in R using rOpenSci
packages, and related packages.

\section{Text mining sources}\label{text-mining-sources}

{[}Describe open source data for text mining that we link to in R, and
those we don't link to in R.{]}

There is increasing open source scientific literature content available
online. However, only a small proportion of scientific journals provide
access to their full content; whereas, many publishers provide open
access to their metadata only (Table 1).

\paragraph{Table 1. Sources of scientific literature, their content type
provided via web services, whether rOpenSci has an R packages for the
service, and where to find the API
documentation.}\label{table-1.-sources-of-scientific-literature-their-content-type-provided-via-web-services-whether-ropensci-has-an-r-packages-for-the-service-and-where-to-find-the-api-documentation.}

\begin{longtable}[]{@{}llll@{}}
\toprule
\begin{minipage}[b]{0.36\columnwidth}\raggedright\strut
Data Provider\strut
\end{minipage} & \begin{minipage}[b]{0.23\columnwidth}\raggedright\strut
Content Type\strut
\end{minipage} & \begin{minipage}[b]{0.10\columnwidth}\raggedright\strut
rOpenSci?\strut
\end{minipage} & \begin{minipage}[b]{0.19\columnwidth}\raggedright\strut
API Documentation\strut
\end{minipage}\tabularnewline
\midrule
\endhead
\begin{minipage}[t]{0.36\columnwidth}\raggedright\strut
Public Library of Science (PLoS)\strut
\end{minipage} & \begin{minipage}[t]{0.23\columnwidth}\raggedright\strut
Full text/altmetrics\strut
\end{minipage} & \begin{minipage}[t]{0.10\columnwidth}\raggedright\strut
rplos\strut
\end{minipage} & \begin{minipage}[t]{0.19\columnwidth}\raggedright\strut
\url{http://api.plos.org/}\strut
\end{minipage}\tabularnewline
\begin{minipage}[t]{0.36\columnwidth}\raggedright\strut
Springer\strut
\end{minipage} & \begin{minipage}[t]{0.23\columnwidth}\raggedright\strut
Full text on OA content\strut
\end{minipage} & \begin{minipage}[t]{0.10\columnwidth}\raggedright\strut
rspringer\strut
\end{minipage} & \begin{minipage}[t]{0.19\columnwidth}\raggedright\strut
\url{http://dev.springer.com/}\strut
\end{minipage}\tabularnewline
\begin{minipage}[t]{0.36\columnwidth}\raggedright\strut
Pensoft\strut
\end{minipage} & \begin{minipage}[t]{0.23\columnwidth}\raggedright\strut
Full text/altmetrics\strut
\end{minipage} & \begin{minipage}[t]{0.10\columnwidth}\raggedright\strut
rpensoft\strut
\end{minipage} & \begin{minipage}[t]{0.19\columnwidth}\raggedright\strut
\url{http://bit.ly/KYP0Zi}\strut
\end{minipage}\tabularnewline
\begin{minipage}[t]{0.36\columnwidth}\raggedright\strut
Nature Publishing Group\strut
\end{minipage} & \begin{minipage}[t]{0.23\columnwidth}\raggedright\strut
Metadata only\strut
\end{minipage} & \begin{minipage}[t]{0.10\columnwidth}\raggedright\strut
No\strut
\end{minipage} & \begin{minipage}[t]{0.19\columnwidth}\raggedright\strut
\url{http://developers.nature.com/}\strut
\end{minipage}\tabularnewline
\begin{minipage}[t]{0.36\columnwidth}\raggedright\strut
Mendeley\strut
\end{minipage} & \begin{minipage}[t]{0.23\columnwidth}\raggedright\strut
Metadata only\strut
\end{minipage} & \begin{minipage}[t]{0.10\columnwidth}\raggedright\strut
rmendeley\strut
\end{minipage} & \begin{minipage}[t]{0.19\columnwidth}\raggedright\strut
\url{http://dev.mendeley.com/}\strut
\end{minipage}\tabularnewline
\begin{minipage}[t]{0.36\columnwidth}\raggedright\strut
DataCite\strut
\end{minipage} & \begin{minipage}[t]{0.23\columnwidth}\raggedright\strut
Metadata only\strut
\end{minipage} & \begin{minipage}[t]{0.10\columnwidth}\raggedright\strut
rdatacite\strut
\end{minipage} & \begin{minipage}[t]{0.19\columnwidth}\raggedright\strut
\url{http://oai.datacite.org/}\strut
\end{minipage}\tabularnewline
\begin{minipage}[t]{0.36\columnwidth}\raggedright\strut
Biodiversity Heritage Library\strut
\end{minipage} & \begin{minipage}[t]{0.23\columnwidth}\raggedright\strut
Full content\strut
\end{minipage} & \begin{minipage}[t]{0.10\columnwidth}\raggedright\strut
rbhl\strut
\end{minipage} & \begin{minipage}[t]{0.19\columnwidth}\raggedright\strut
\url{http://bit.ly/KYQ1Rd}\strut
\end{minipage}\tabularnewline
\begin{minipage}[t]{0.36\columnwidth}\raggedright\strut
Scopus (Elsevier)\strut
\end{minipage} & \begin{minipage}[t]{0.23\columnwidth}\raggedright\strut
Metadata only ????\strut
\end{minipage} & \begin{minipage}[t]{0.10\columnwidth}\raggedright\strut
No\strut
\end{minipage} & \begin{minipage}[t]{0.19\columnwidth}\raggedright\strut
\url{http://bit.ly/J9S616}\strut
\end{minipage}\tabularnewline
\bottomrule
\end{longtable}

\paragraph{Table 2. Tools for text mining science
literature.}\label{table-2.-tools-for-text-mining-science-literature.}

\begin{longtable}[]{@{}llll@{}}
\toprule
Tool & Platform & Services & URL\tabularnewline
\midrule
\endhead
PMC Miner & Python & PubMed & \url{http://bit.ly/L29ekY}\tabularnewline
rOpenSci & R & See Table 1 & \url{http://ropensci.org/}\tabularnewline
PMC Miner & Python & PubMed & \url{http://bit.ly/L29ekY}\tabularnewline
PMC Miner & Python & PubMed & \url{http://bit.ly/L29ekY}\tabularnewline
PMC Miner & Python & PubMed & \url{http://bit.ly/L29ekY}\tabularnewline
PMC Miner & Python & PubMed & \url{http://bit.ly/L29ekY}\tabularnewline
PMC Miner & Python & PubMed & \url{http://bit.ly/L29ekY}\tabularnewline
PMC Miner & Python & PubMed & \url{http://bit.ly/L29ekY}\tabularnewline
PMC Miner & Python & PubMed & \url{http://bit.ly/L29ekY}\tabularnewline
PMC Miner & Python & PubMed & \url{http://bit.ly/L29ekY}\tabularnewline
PMC Miner & Python & PubMed & \url{http://bit.ly/L29ekY}\tabularnewline
PMC Miner & Python & PubMed & \url{http://bit.ly/L29ekY}\tabularnewline
\bottomrule
\end{longtable}

\section{How to text mine from R: Three case
studies}\label{how-to-text-mine-from-r-three-case-studies}

\subsubsection{Case study 1}\label{case-study-1}

\subsubsection{Case study 2}\label{case-study-2}

\subsubsection{Case study 3}\label{case-study-3}

\section{Conclusions and future
directions}\label{conclusions-and-future-directions}

\section{Acknowledgments}\label{acknowledgments}

\section{Data Accessibility}\label{data-accessibility}

All scripts and data used in this paper can be found in the permanent
data archive Zenodo under the digital object identifier (DOI). This DOI
corresponds to a snapshot of the GitHub repository at
\url{https://github.com/ropensci/textmine}. Software can be found at
\url{https://github.com/ropensci/xxx}, xxxx, all under MIT licenses.

\section{References}\label{references}

\end{document}


